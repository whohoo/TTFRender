\documentclass[a4paper,12pt]{report}

\usepackage {CJKutf8}
%\usepackage{listings}     % code highlight      %
\usepackage{indentfirst}   % 中文首段缩进        %
\usepackage[colorlinks,linkcolor=blue,anchorcolor=blue,citecolor=blue]{hyperref} % 超链接支持 %

% 设置中文首段缩进策略 %
\setlength{\parindent}{2em}
\setlength\textwidth{357.0pt}

\makeatletter
\def\enumerate{%
 \ifnum \@enumdepth >\thr@@\@toodeep\else
   \advance\@enumdepth\@ne
   \edef\@enumctr{enum\romannumeral\the\@enumdepth}%
     \expandafter
     \list
       \csname label\@enumctr\endcsname
       {\usecounter\@enumctr\def\makelabel##1{\hss\llap{##1}}%
         %\addtolength{\leftmargin}{5pt}
         \addtolength{\parsep}{10pt}
         %\addtolength{\rightmargin}{20pt} %%%%
         \addtolength{\listparindent}{10pt} %%%%
         \addtolength{\itemsep}{-8pt} %%%%
         \addtolength{\topsep}{-4pt}} %%%%
 \fi}
\makeatother

\begin{CJK*}{UTF8}{gkai}
\begin{document}

\author{xiaomo(wxm4ever@gmail.com)}
\title{\Large{WebFont \&\& WebFont Loader 中文解决方案}}
\date{\today}

\clearpage
\thispagestyle{empty}

\maketitle
\begin{abstract}

使用网络字体的好处是不言而喻的,它可以让我们更方便的实现更丰富的页面效果。国外的技术人员,一直在不断的为让web-font变为现实而努力,从sIFR到typeface再到typeselect,以及cufon和\\fontue,不同的尝试和解决方案不断涌现,各大浏览器纷纷开始支持\emph{@font-face}属性,同时类似typekit之类的收费和免费网络字体也开始不断出现。这让我们对此充满的希望。但是反观国内,我们却只能望洋兴叹,汉字的网络应用一直没有进展。本方案可较有效的解决中文在\underline{WebFont}应用中的瓶颈问题。

\end{abstract}

\newpage
\setcounter{page}{1} 

\chapter{现状}

由 css3 引入的 \href{http://www.w3schools.com/cssref/css3_pr_font-face_rule.asp}{font-face} 可以方便的使用网络字体。使用网络字体的好处是不言而喻的,它可以有效解决用户终端字库匮乏的现状,并且有效减少服务器、网络压力。\href{http://www.google.com/webfonts/}{Google WebFonts} 又提供了很多优美的在线字体供我们免费使用,唯一遗憾的事情就是 google 字库里没有中文。

中文网络字体的实现有很多难以跨越的瓶颈:

\begin{enumerate}
    \item 字体太大,在当前国内的网络带宽环境下,使用中文网络字体不太现实
    \item 版权问题,绝大部分优秀的中文字体,都不是免费的
    \item 商业模式陈旧,没有与时俱进
\end{enumerate}

国内的商业字体厂商的思维还是停留在平面和印刷上,他们并没有重视或者发现网络这块市场,以至于多年来,除了开源的文泉驿,我们没有见到有太多字体创新,更没有看到各大字体厂商有向网络字体方面的任何努力。

\href{https://developers.google.com/webfonts/docs/webfont_loader}{WebFont Loader} 是一个很有用的 javascript 库,通过它我们可以很容易的控制在不同加载状态下的字体样式。它是 google 和 Typekit 联合开发的,在 \href{https://github.com/typekit/webfontloader}{github} 上可以找到它的源码。(\emph{GitHub 上的文档比 google 给的更加详细哦!~ :)})


\chapter{解决方案}

基本思路就是:

\begin{enumerate}

\item 缩小字库

中文字库一般很大,所以我们需要对字库进行适当缩小。这里使用 python 来实现,依赖于 \href{http://fontforge.sourceforge.net/}{FontForge} 模块。这里我封装了一个模块 \emph{TTFRender}。下面是一个简单的使用范例:

\begin{verbatim}

#-*- coding: utf-8 -*-

from TTFRender import generate, get_token_list


str = """
嗯,除了希望,我没有任何办法促进中文字体的网络应用……

博大精深的中华文化,很大一部分是体现在汉字上的,而在互联网对社会生活影响越来越深刻的今天,汉字在网络上的应用却依然那么困难和苍白。

Google之所以令人尊敬,不仅仅是因为它在创新,更是因为它也在普及创新。

希望国内的字体厂商和浏览器厂商,能够真正的为中文字体的网络应用做出些努力和创新,不要固步自封,固守已经过时的商业模式。

不过,或许他们也无能为力,或许我们只有等待——等待网速普及到100MB以上、等待国外出现先进的技术、等待….——到时候,或许字体厂商会沦为坚强的维权者….

"""

token_list = get_token_list(str)
ttf_file   = "t2.ttf"

generate(token_list, "kai", ttf_file)

\end{verbatim}

这里用到了 fontforge 模块,具体的 API 可以查看 \href{http://fontforge.sourceforge.net/python.html}{API Reference}。这里讲述下 parser 的实现。


\begin{verbatim}

#-*- coding: utf-8 -*-

"""
Parse the TTF FILE && Generate a New TTF FILE
"""

try:
    import fontforge
except ImportError:
    raise ImportError("Need Module fontforge!")


from helper import get_ttf_file
from helper import _unicode

def generate(token_list, font_name, filename):
    """ Generate the TTF FILE of Specified Font """
    font_file = get_ttf_file(font_name)
    fnt = fontforge.open(font_file)
    for w in fnt.glyphs():
        if w.unicode<0 or \
                (unichr(w.unicode) not in map(lambda x:_unicode(x), token_list)):
            fnt.removeGlyph(w)
        else:
            print unichr(w.unicode)

    fnt.generate(filename)
    fnt.close()

\end{verbatim}

ttf 字体中包含着所有字的 glyphs 信息,所以我们需要把用到的字的 glyphs 筛选出来形成新的 ttf 文件,从而对字体文件缩小,这是使中文 WebFont 可用的基础。

\item 通过 @font-face 使用

生成较小的字库之后,我们便可以通过定义 @font-face 来使用。

\begin{verbatim}

<!doctype html>
<html lang="en">
<head>
  <meta charset="utf-8">
  <title>WebFont Test</title>
  <style type="text/css">
 @font-face {
   font-family: 'WuJiaRui';
       font-weight: 400;
         src: url('new.ttf');
         }
  p { font-size:20pt; }
  </style>
</head>
<body>
不是用 WebFont:<br/>
一二三<br/>

<hr />

部分字使用 WebFont:<br />
<p style="font-family:WuJiaRui;">一</p>
<p style="font-family:WuJiaRui">二</p>
<p style="font-family:WuJiaRui">三</p>
<p style="font-family:WuJiaRui">测试</p>
</body>
</html>

\end{verbatim}

字体文件在加载状态中,不同浏览器对页面的输出不一样。具体的可参看 \href{https://developers.google.com/webfonts/docs/technical_considerations}{technical considerations}。

\item 使用 WebFont Loader[optional]

使用 WebFont Loader 可以有效的控制在不同加载状态下的字体显示。这在字体文件很大或者网络传输很不稳定的环境下是非常有用的。

\begin{verbatim}

<!doctype html>
<html lang="en">
<head>
  <meta charset="utf-8">
  <title>WebFontLoader Test</title>
  <script type="text/javascript" src="https://www.google.com/jsapi"></script>
  <script type="text/javascript">
    google.load("webfont", "1");

    google.setOnLoadCallback(function() {
        WebFont.load({
            /*
            custom: { 
                families:[ 'WuJiaRui' ],
                urls:[ "http://tmpdemo.sinaapp.com/douban/webfont/wu.css" ]
            }
            */
            google: { families:['Tangerine']}
        });
  </script>
  <style type="text/css">

.wf-loading  p { font-family: serif; }
.wf-inactive p { font-family: WuJiaRui; }
.wf-active   p { font-family: Tangerine; }

  </style>
</head>
<body>
ABCDEFG一二三
<p>ABC</p>
<p>一</p>
<p>二</p>
<p>三</p>
<p>测试</p>
</body>
</html>

\end{verbatim}

其中 \emph{wf-loading} 控制在加载状态中的字体样式,\emph{wf-active} 控制在加载完毕状态下的字体样式,\emph{wf-inactive} 控制在加载错误下的字体样式。

\item 其他

中文字体在缩小字库情况下仍然很大,这时候需要采取一定的手段来提升 UE。这里我们使用的策略是把字体文件继续分割成更小的字体文件。然后通过 WebFontLoader 逐个载入。当然,我们也可以把常用字的字库长期缓存在客户端,然后每次只通过 WebFontLoader 来加载稀有字。

\begin{verbatim}

<!doctype html>
<html lang="en">
<head>
  <meta charset="utf-8">
  <title>WebFontLoader Test</title>
  <script type="text/javascript" src="http://code.jquery.com/jquery-1.7.2.min.js"></script>
  <script type="text/javascript">
  $(function(){
    /*
    alert("JQuery OK!");
    $("button").click(function(){
      $("p").css("font-family", "B,A");
    });
    */
  });
  </script>
  <script type="text/javascript">
WebFontConfig = {
  custom: { families: [ 'A','B' ], urls: ['A.css', 'B.css'] },
  //google:{families:['Creepster']},
  fontactive: function(fontname, n4){
    var font_name = $("p").css("font-family");
    $("p").css("font-family", fontname + "," + font_name);
  }
};
(function() {
   var wf = document.createElement('script');
   wf.src = ('https:' == document.location.protocol ? 'https' : 'http') +
                  '://ajax.googleapis.com/ajax/libs/webfont/1/webfont.js';
   wf.type = 'text/javascript';
   wf.async = 'true';
   var s = document.getElementsByTagName('script')[0];
   s.parentNode.insertBefore(wf, s);
})();
  </script>
  <style type="text/css">

p { font-size:20pt; }

  </style>
</head>
<body>
<p>ABCDEFG一二三</p>
<p>ABC</p>
<p>一</p>
<p>二</p>
<p>三</p>
<p>测试</p>
<button>载入字体二</button>
<p>使用网络字体的好处是不言而喻的,它可以让我们更方便的实现更丰富的页面效果。国外的技术人员,一直在不断的为让web-font变为现实而努力,从sIFR到typeface再到typeselect,以及cufon和fontue,不同的尝试和解决方案不断涌现,各大浏览器纷纷开始支持@web-font属性,同时类似typekit之类的收费和免费网络字体也开始不断出现。这让我们对此充满的希望。

但是反观国内,我们却只能望洋兴叹,汉字的网络应用一直没有进展。

其实道理很简单,中文网络字体的实现有很多难以跨越的瓶颈:

    字体太大,在当前国内的网络带宽环境下,使用中文网络字体不太现实;
    版权问题,绝大部分优秀的中文字体,都不是免费的;
    商业模式陈旧,没有与时俱进。

国内的商业字体厂商的思维还是停留在平面和印刷上,他们并没有重视或者发现网络这块市场,以至于多年来,除了开源的文泉驿,我们没有见到有太多字体创新,更没有看到各大字体厂商有向网络字体方面的任何努力。</p>


<p>尊敬尊敬尊敬尊敬尊敬尊敬尊敬尊敬尊敬尊敬</p>
</body>

\end{verbatim}

上面的代码将字体文件又细分成更多子块字体文件。然后通过 WebFont Loader 的 fontactive 事件来动态加载使用。

\item 兼容性说明

WebFont \&\& WebFontLoader 归根结底是采用 js 来控制 font-face。所以主要需要考虑各个浏览器对 font-face 的支持以及所支持的字体文件格式。

首先,我们可以从 \href{http://www.w3schools.com/cssref/css3_browsersupport.asp}{w3schools} 上看到 css3 font-face 对各个浏览器的支持。这里需要特殊说明的有:

  \begin{enumerate}

  \item IE
  IE4 开始就可以使用 font-face 了。但是 IE 的 font-face 只支持 EOT 格式的字体文件。

  \item Safari
  Safari 可以支持 TTF/OTF 的 font-face。但是在移动端 ios4.1 以前的版本需要使用 SVG 格式的字体文件。

  \item Chrome.Firefox.Opera
  均可以支持 TTF/OTF 的 font-face。

  \end{enumerate}

\end{enumerate}
\end{document}
\end{CJK*}
